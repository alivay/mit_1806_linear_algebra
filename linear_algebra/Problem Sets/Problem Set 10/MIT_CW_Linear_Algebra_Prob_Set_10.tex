\documentclass[a4paper,11pt]{article}

%Headers
\usepackage[dvips]{graphicx}    %package that does pdfs
\usepackage{color}              %this needs to be here also
\usepackage{ulem}
\usepackage{amsmath}
\usepackage{pgfplots}
\usepackage{adjustbox}
\usepackage{graphicx}
\usepackage{enumitem}
\usepackage{listings}
\usepackage{tikz}
\usepackage{fancyvrb}
\usepackage{upquote}

\newcommand*\circled[1]{\tikz[baseline=(char.base)]{
             \node[shape=circle,draw,inner sep=2pt] (char) {#1};}}
\newcommand{\mybf}[1]{\boldsymbol{#1}}
\newcommand{\norm}[1]{\lvert\lvert #1 \rvert\rvert}
\newcommand{\autour}[1]{\tikz[baseline=(X.base)]\node [draw=black,fill=white!40,semithick,rectangle,inner sep=2pt, rounded corners=3pt] (X) {#1};}
\newcommand{\?}{\stackrel{?}{=}}

\title{%
	Problem Set 10\\
	\large MIT CW Linear Algebra (18.06)
}
\author{Aviel Livay}
\date{\today}

\begin{document}
\maketitle

\subsection*{Former Section 6.6, question 12}
\begin{align*}
JM = 
\begin{bmatrix}
m_{2,1} & m_{2,2}, & m_{2,3} & m_{2,4} \\
0 & 0, & 0 & 0 \\
m_{4,1} & m_{4,2}, & m_{4,3} & m_{4,4} \\
0 & 0, & 0 & 0 \\
\end{bmatrix}\\\\
MK = 
\begin{bmatrix}
0 & m_{1,1}, & m_{1,2} & 0 \\
0 & m_{2,1}, & m_{2,2} & 0 \\
0 & m_{3,1}, & m_{3,2} & 0 \\
0 & m_{4,1}, & m_{4,2} & 0 \\
\end{bmatrix}
\end{align*}
In order for the matrices to be the same we need:
\begin{enumerate}
\item $m_{2,1}=0$
\item $m_{2,2}=m_{1,1}=a$, 
\item $m_{2,3}=m_{1,2}=b$, 
\item $m_{2,4}=0$, 
\item $m_{4,1}=0$, 
\item $m_{4,2}=m_{3,1}=c$,
\item $m_{4,3}=m_{3,2}=d$
\item $m_{4,4}=0$ 
\end{enumerate}
So we end up with
\begin{align*}
M = 
\begin{bmatrix}
a & b & m_{1,3} & m_{1,4} \\
0 & a & b & 0 \\
c & d & m_{3,3} & m_{3,4} \\
0 & c & d & 0 \\
\end{bmatrix}
\end{align*}
\subsection*{Former Section 6.6, question 14}
We know that 
\begin{align*}
A=X^{-1}\Lambda{X}
\end{align*}
Where X contains the eigenvectors of A
So we can write it as 
\begin{align*}
\Lambda = X{A}X^{-1}
\end{align*}
If we multiply both sides by $Y^{-1}$ and $Y$ where $Y$ is the matrix containing the eigenvectors of $A$ then we get
\begin{align*}
Y^{-1}\Lambda{Y} &= Y^{-1}X{A}X^{-1}{Y} \\
A^T &= (Y^{-1})X{A}(X^{-1}{Y})\\
A^T &= (X^{-1}{Y})^{-1}{A}(X^{-1}{Y})\\
\end{align*}
\subsection*{Former Section 6.6, question 20}
\begin{enumerate}[label=\alph*]
\item 
If $A$ is similar to $B$ then there exist $M$ such that
\begin{align*}
A=M^{-1}B{M}
\end{align*}
Now if we square both sides then we get - 
\begin{align*}
A^2&=M^{-1}B{M}M^{-1}B{M} \\
A^2&=M^{-1}B^2{M} \\
\end{align*}
\item
Suppose $A$ is similar to $-B$. So there's $M$ such that $M^{-1}A{M}=-B$. In that case $A$ cannot be similar to $B$ because their eigenvalues are not the same, they are negative to each other. So if we square both sides, still we can get that $A^2$ is similar to $B^2$
\item
There's a matrix $M_A$ such that $M_A^{-1}{A}M_A=\Lambda$ and then there's a matrix $M_B$ such that  ${B}=M_B\Lambda{M_B}^{-1}$ so ${B}=M_BM_A^{-1}{A}M_A{M_B}^{-1}$ And we get that $B=(M_A{M_B}^{-1})^{-1}{A}(M_A{M_B}^{-1})$
\end{enumerate}
\subsection*{Section 7.3, question 11 (Former Section 6.7, question 4)}
\begin{align*}
A^T{A}=
\begin{bmatrix}
1 & 1 \\
1 & 0 \\
\end{bmatrix}
\begin{bmatrix}
1 & 1 \\
1 & 0 \\
\end{bmatrix}
=
\begin{bmatrix}
2 & 1 \\
1 & 1 \\
\end{bmatrix}
\end{align*}
The 
\end{document}