\documentclass[a4paper,11pt]{article}

%Headers
\usepackage[dvips]{graphicx}    %package that does pdfs
\usepackage{color}              %this needs to be here also
\usepackage{ulem}
\usepackage{amsmath}
\usepackage{pgfplots}
\usepackage{adjustbox}
\usepackage{graphicx}
\usepackage{enumitem}
\usepackage{listings}
\usepackage{tikz}
\usepackage{fancyvrb}
\usepackage{upquote}

\newcommand*\circled[1]{\tikz[baseline=(char.base)]{
             \node[shape=circle,draw,inner sep=2pt] (char) {#1};}}
\newcommand{\mybf}[1]{\boldsymbol{#1}}
\newcommand{\norm}[1]{\lvert\lvert #1 \rvert\rvert}

\title{%
	Problem Set 5\\
	\large MIT CW Linear Algebra (18.06)
}
\author{Aviel Livay}
\date{\today}

\begin{document}
\maketitle

\subsection*{Worked example - 4.1A}
\begin{enumerate}[label=\alph*]
\item spaces with 0, 1, 2 or 3 dimensions
\item 3
\item 6x9
\item 6x9
\end{enumerate}
\subsection*{Section 4, question 7}
We start with the matrix:
\begin{align*}
A = 
\begin{bmatrix}
1 & -1 & 0 \\
0 & 1 & -1\\
1 & 0 & -1\\
\end{bmatrix}\\
\end{align*}
We are actually looking for a vector that performs linear combination on the rows of $A$ and sends them to 0.
Such a vector is in the left null space.
For convenience we shall be looking for it in the column space of $A^T$.
\begin{align*}
A^T = 
\begin{bmatrix}
1 & 0 & 1 \\
-1 & 1 & 0\\
0 & -1 & -1\\
\end{bmatrix}\\
\end{align*}
The special solution for this matrix is $C*(-1,-1,1)$. Dot product with $\mybf{b}=(1,1,1)$ give $-C$ so essentially $C=-1$ and the vector that we are looking for is $(1,1,-1)$
It means that $y_1=1$, $y_2=1$ and $y_3=-1$ and indeed if we combine the rows of A this way we get $0$ and if we combine the solution $(1,1,1)$ we get $1$. 
\subsection*{Section 4.1, question 9}
If $A^TA\mybf{x} = 0$ then $A\mybf{x} = 0$. Reason: $A\mybf{x}$ is in the null space of $A^T$ and also in the
column space of A and those spaces are orthogonal. Conclusion: $A^TA$ has the same null space as A.
\subsection*{Section 4.1, question 31}
$N$ is the null space of $A$. Thus $N'$ is the left null space of $A^T$. $B$ is the null space of $N'$ which means it's complementing and orthogonal to this space, which means it is the row space of A.  
To illustrate this, I took as requested a random matrix of dimensions 6x12. I created N, and calculated B as the null space of N'. Then I added B' at the bottom of A. Each line in that matrix is a linear combination of the rows of A, so when reducing to echelon form I was left with 6 pivots.
\begin{Verbatim}[fontsize=\tiny]
>> A=rand(6,12)
A =

   0.444654   0.091695   0.163208   0.136976   0.855099   0.148510   0.993936   0.275570   0.731713   0.286653   0.517904   0.204220
   0.512112   0.439665   0.078789   0.396367   0.468470   0.963250   0.813834   0.190640   0.440366   0.565692   0.987456   0.566036
   0.998993   0.976260   0.485692   0.876608   0.981196   0.668222   0.295224   0.990402   0.165956   0.955917   0.183341   0.446471
   0.994070   0.022294   0.463387   0.266155   0.545232   0.402779   0.127522   0.043596   0.214838   0.921090   0.597919   0.032974
   0.184608   0.280592   0.649525   0.079031   0.101671   0.504168   0.939635   0.908437   0.063399   0.492408   0.537304   0.313639
   0.923165   0.350907   0.011296   0.721968   0.287685   0.792693   0.830567   0.896162   0.752956   0.114624   0.932165   0.850104

>> N=null(A)
N =

   2.3285e-01  -2.5112e-02  -2.3838e-01  -2.5244e-01  -4.0701e-01  -8.9754e-02
   5.7609e-01  -1.9746e-02   2.5389e-01  -1.9488e-01  -7.2454e-02  -2.7393e-01
  -1.4525e-02  -4.5621e-01   3.0811e-01  -4.1418e-01   2.4363e-01   2.0469e-01
  -2.5006e-01  -5.8525e-01  -1.4046e-01   3.2483e-01   3.6212e-01  -2.2300e-01
  -1.9095e-01   2.3872e-01  -2.2861e-01  -2.3038e-01   2.9013e-01   2.1044e-01
  -5.8847e-01   1.0082e-01   7.2895e-02  -3.3836e-01  -2.4707e-01  -4.6235e-02
   5.4514e-02  -3.1362e-01  -3.2729e-01   9.9099e-02  -3.2287e-01  -1.5376e-01
  -1.4713e-01   4.5729e-01  -3.3797e-03   1.0940e-01   1.0045e-01  -9.0621e-02
  -1.1140e-01  -4.5911e-03   7.5828e-01   1.0279e-01  -9.3115e-02  -5.1709e-02
   1.1881e-02   1.1536e-01   1.4364e-01   6.4010e-01  -1.6644e-01   8.1389e-02
   2.9362e-01   2.3279e-01  -5.1970e-02  -1.6616e-02   5.8446e-01  -1.7447e-01
   2.1218e-01  -7.8673e-02  -3.2026e-02   1.0640e-01  -3.3075e-02   8.4099e-01

>> B=null(N')
B =

   1.3155e-01  -2.8420e-01  -1.9117e-01  -3.0052e-01  -5.3832e-01   3.6900e-01
  -2.5995e-01   3.6115e-01   1.7252e-01   4.3873e-01  -6.1616e-02   2.4690e-01
   3.4159e-01   1.2410e-01  -4.8918e-01   1.4939e-01  -1.2211e-01  -1.2377e-01
  -2.1037e-01   1.2095e-01   1.0789e-01  -6.9721e-02  -2.6086e-01   3.8134e-01
   2.0330e-01  -7.0396e-02   4.5417e-01   4.2852e-01  -4.8419e-01  -4.6117e-02
  -9.7087e-02  -3.2416e-02  -9.1174e-02   2.6471e-01   3.0250e-01   5.2934e-01
   6.5993e-01   2.1804e-01   2.6141e-01   1.4225e-01   2.8659e-01   9.1339e-03
   2.3030e-01   7.1592e-01  -2.2525e-01  -2.4922e-01  -1.9917e-01   1.4443e-01
   2.9842e-01  -1.3771e-01   4.2898e-01  -2.5810e-01  -1.3539e-01   1.1697e-01
   1.4957e-01  -1.9551e-01  -3.7410e-01   5.2099e-01  -1.9372e-01   1.1123e-01
   3.1763e-01  -3.2949e-01  -8.6204e-02  -5.2627e-02   3.2133e-01   4.0210e-01
  -5.0522e-02   1.6411e-01   1.3302e-01  -1.1475e-01   1.2884e-01   3.8876e-01

>> C =[A;B']
C =

   4.4465e-01   9.1695e-02   1.6321e-01   1.3698e-01   8.5510e-01   1.4851e-01   9.9394e-01   2.7557e-01   7.3171e-01   2.8665e-01   5.1790e-01   2.0422e-01
   5.1211e-01   4.3967e-01   7.8789e-02   3.9637e-01   4.6847e-01   9.6325e-01   8.1383e-01   1.9064e-01   4.4037e-01   5.6569e-01   9.8746e-01   5.6604e-01
   9.9899e-01   9.7626e-01   4.8569e-01   8.7661e-01   9.8120e-01   6.6822e-01   2.9522e-01   9.9040e-01   1.6596e-01   9.5592e-01   1.8334e-01   4.4647e-01
   9.9407e-01   2.2294e-02   4.6339e-01   2.6616e-01   5.4523e-01   4.0278e-01   1.2752e-01   4.3596e-02   2.1484e-01   9.2109e-01   5.9792e-01   3.2974e-02
   1.8461e-01   2.8059e-01   6.4953e-01   7.9031e-02   1.0167e-01   5.0417e-01   9.3964e-01   9.0844e-01   6.3399e-02   4.9241e-01   5.3730e-01   3.1364e-01
   9.2317e-01   3.5091e-01   1.1296e-02   7.2197e-01   2.8769e-01   7.9269e-01   8.3057e-01   8.9616e-01   7.5296e-01   1.1462e-01   9.3216e-01   8.5010e-01
   1.3155e-01  -2.5995e-01   3.4159e-01  -2.1037e-01   2.0330e-01  -9.7087e-02   6.5993e-01   2.3030e-01   2.9842e-01   1.4957e-01   3.1763e-01  -5.0522e-02
  -2.8420e-01   3.6115e-01   1.2410e-01   1.2095e-01  -7.0396e-02  -3.2416e-02   2.1804e-01   7.1592e-01  -1.3771e-01  -1.9551e-01  -3.2949e-01   1.6411e-01
  -1.9117e-01   1.7252e-01  -4.8918e-01   1.0789e-01   4.5417e-01  -9.1174e-02   2.6141e-01  -2.2525e-01   4.2898e-01  -3.7410e-01  -8.6204e-02   1.3302e-01
  -3.0052e-01   4.3873e-01   1.4939e-01  -6.9721e-02   4.2852e-01   2.6471e-01   1.4225e-01  -2.4922e-01  -2.5810e-01   5.2099e-01  -5.2627e-02  -1.1475e-01
  -5.3832e-01  -6.1616e-02  -1.2211e-01  -2.6086e-01  -4.8419e-01   3.0250e-01   2.8659e-01  -1.9917e-01  -1.3539e-01  -1.9372e-01   3.2133e-01   1.2884e-01
   3.6900e-01   2.4690e-01  -1.2377e-01   3.8134e-01  -4.6117e-02   5.2934e-01   9.1339e-03   1.4443e-01   1.1697e-01   1.1123e-01   4.0210e-01   3.8876e-01

>> rref(C)
ans =

   1.0000        0        0        0        0        0  -5.5763  -4.0946  -2.8154   2.7198  -1.4730  -2.0959
        0   1.0000        0        0        0        0  -6.9244  -4.5585  -4.2527   3.1955  -2.8319  -2.5897
        0        0   1.0000        0        0        0   3.0096   2.9984   1.1204  -0.6444   0.7776   0.9225
        0        0        0   1.0000        0        0   8.3193   7.5985   4.9842  -4.4709   2.4087   3.8438
        0        0        0        0   1.0000        0   2.5387   1.0515   1.5938  -0.5557   0.8582   0.6570
        0        0        0        0        0   1.0000   2.0660   0.5722   0.9774  -0.1545   1.6287   0.9073
        0        0        0        0        0        0        0        0        0        0        0        0
        0        0        0        0        0        0        0        0        0        0        0        0
        0        0        0        0        0        0        0        0        0        0        0        0
        0        0        0        0        0        0        0        0        0        0        0        0
        0        0        0        0        0        0        0        0        0        0        0        0
        0        0        0        0        0        0        0        0        0        0        0        0

>>
\end{Verbatim}
\subsection*{Section 4.1, question 32}
\begin {enumerate} [label=\alph*]
\item $r \cdot n = 0$ and $c \cdot l = 0$. Also their dimensions should add up to 2. In this case dimensions should be 1 for all vectors. 
\item the matrix $\mybf{c} * \mybf{r}^T $ should do the trick because
\begin{align*}
A*\mybf{n}=
\begin{bmatrix}
c_1r_1 & c_1r_2 \\
c_2r_1 & c_2r_2 \\
\end{bmatrix}
*
\begin{bmatrix}
n_1 \\
n_2 \\
\end{bmatrix}
= 0
\end{align*}
and also 
\begin{align*}
A^T*\mybf{l}=
\begin{bmatrix}
c_1r_1 & c_2r_1 \\
c_1r_2 & c_2r_2 \\
\end{bmatrix}
*
\begin{bmatrix}
l_1 \\
l_2 \\
\end{bmatrix}
= 0
\end{align*}
\end {enumerate}
\subsection*{Section 4.1, question 33}
\begin {enumerate} [label=\alph*]
\item $r \cdot n = 0$ and $c \cdot l = 0$. Also their dimensions should add up to 4. The dimension of ${r_i}$ can't exceed 2 because there are only two vectors. It can't be less than 2 because then the left null space shall have to have more than 2 such vectors. So that's why the dimension of $r_1$ and $r_2$ = 2 and actually the dimension of all four basis formed by the pairs is 2.
\item One such A matrix that comes to mind is:
\begin{align*}
A =
\begin{bmatrix}
\mybf{c_1} & \mybf{c_2} & 0 & 0 \\
\end{bmatrix}
*
\begin{bmatrix}
\mybf{r_1^T}  \\
\mybf{r_2^T}  \\
0 \\
0 \\
\end{bmatrix}
\end{align*}

Because
\begin{align*}
\mybf{l_i^T}*A =
\mybf{l_i^T} *
\begin{bmatrix}
\mybf{c_1} & \mybf{c_2} & 0 & 0 \\
\end{bmatrix}
*
\begin{bmatrix}
\mybf{r_1^T}  \\
\mybf{r_2^T}  \\
0 \\
0 \\
\end{bmatrix}
= 0
\end{align*}
And also - 
\begin{align*}
A * \mybf{n_i}=
\begin{bmatrix}
\mybf{c_1} & \mybf{c_2} & 0 & 0 \\
\end{bmatrix}
*
\begin{bmatrix}
\mybf{r_1^T}  \\
\mybf{r_2^T}  \\
0 \\
0 \\
\end{bmatrix}
*
\mybf{l_i}
= 0
\end{align*}
\end {enumerate}
\subsection*{Section 4.2, question 13}
We know the following:
\begin{align*}
A=
\begin{bmatrix}
1 & 0 & 0 \\
0 & 1 & 0 \\
0 & 0 & 1 \\
0 & 0 & 0 \\
\end{bmatrix},
\mybf{b}=
\begin{bmatrix}
1  \\
2  \\
3  \\
4  \\
\end{bmatrix}\\
\end{align*}
We know that $\hat{x} = A(A^TA)^{-1}A^T\mybf{b}=(1,2,3,0)$
The projection matrix is 4x4 because it takes a vector of size 4 ($\mybf{b}$) and project it onto the columns space that has size 4 as well.
\begin{align*}
P=A(A^TA)^{-1}A^T = 
\begin{bmatrix}
1 & 0 & 0 & 0 \\
0 & 1 & 0 & 0\\
0 & 0 & 1 & 0\\
0 & 0 & 0 & 0\\
\end{bmatrix}
\end{align*}
But of course since this is a quick and recommended question then we can solve it without all of these equations... - we just take $(1,2,3,4)$ and of course the closest vector in the column space is $(1,2,3,0)$. And $P$ should just copy the first 3 and blank the fourth, hence the $P$ that we calculated.
\subsection*{Section 4.2, question 16}
We actually want to project $\mybf{b} = (2,1,1)$ onto the column space that is span by the vectors $(1,2,-1)$ and $(1,0,1)$. So what we are looking for is $\hat{x}=(A^TA)^{-1}A^T\mybf{b}$ where 
\begin{align*}
A = 
\begin{bmatrix}
1 & 1  \\
2 & 0  \\
-1 & 1 \\
\end{bmatrix}
\end{align*}
So,
$\hat{x}=(A^TA)^{-1}A^T\mybf{b} = (0.5, 1.5)$
\subsection*{Section 4.2, question 17}
\begin{align*}
(I-P)^2=I^2-IP-PI+P^2=I-P+P^2-P=I-P
\end{align*}
Which proves that $(I-P)^2=I-P$.
$P^2=P$ implies that if $\mybf{x}$ is in the column space of $A$ then $P\mybf{x}$ always stays in the column space of $A$. That is $P\mybf{x}=\mybf{x}$. If so then for a $\mybf{x}$ that's in the column space of $A$ we have $(I-P)\mybf{x}=I\mybf{x}-P\mybf{x}=\mybf{x}-\mybf{x}=\mybf{0}$. Which means $I-P$ is orthogonal to any such $\mybf{x}$ that's in the column space of $A$. Or in other words $I-P$ is in the column space of $A^T$ or yet in other words $I-P$ is in the left column space of $A$.
\subsection*{Section 4.2, question 30}
\begin{align*}
Pc=
\begin{bmatrix}
\frac{9}{25} & \frac{12}{25} \\
\frac{12}{25} & \frac{16}{25} \\
\end{bmatrix}
\end{align*}
If you multiply $Pc$ with any vector that belongs to the column space of $A$ then you will end up with the same vector. So no wonder that specifically $Pc*A=A$.
\begin{align*}
Pr=
\begin{bmatrix}
\frac{1}{9} & \frac{2}{9} & \frac{2}{9}\\
\frac{2}{9} & \frac{4}{9} & \frac{4}{9}\\
\frac{2}{9} & \frac{4}{9} & \frac{4}{9}\\
\end{bmatrix}
\end{align*}
If you multiply $Pr$ with any vector that belongs to the column space of $A^T$ then you will end up with the same vector. That is $Pr*A^T=A^T$. If we transpose both sides then we get $A*Pr=A$.
So from the above it's not wonder that multiplying $A$ by $Pc$ from the left yields $A$ and then continuing and multiplying that $A$ result by $Pr$ on the right - leaves us with the original $A$.
\subsection*{Section 4.2, question 31}
I would check that $(b-p) \cdot p = 0$.
\subsection*{Section 4.2, question 34}
I don't know.
\subsection*{Section 10.1, question 13 (former(Section 8.1, question 13)}
\begin{align*}
A = 
\begin{bmatrix}
-1 & 1  & 0  & 0 \\
-1 & 0  & 1  & 0 \\
0  & -1 & 1  & 0 \\
0  & -1 & 0  & 1 \\
0  & 0  & -1 & 1 \\
\end{bmatrix},
C = 
\begin{bmatrix}
2 & 0  & 0  & 0 & 0 \\
0 & 2  & 0  & 0 & 0 \\
0 & 0  & 3  & 0 & 0 \\
0 & 0  & 0  & 3 & 0 \\
0 & 0  & 0  & 0 & 3 \\
\end{bmatrix},
\mybf{f} = 
\begin{bmatrix}
1  \\
0 \\
0 \\
-1 \\
\end{bmatrix}
\end{align*}
$A^TCA\mybf{x}=\mybf{f}$
This $A^TCA$ matrix has a rank 3 so it's not invertible, so solving by inversion doesn't work.
\begin{align*}
A^T*C*A=
\begin{bmatrix}
4 & -2 & -2 & 0 \\
-2 & 8 & -3 & -3 \\
-2 & -3 & 8 & -3 \\
0 & -3 & -3 & 6 \\
\end{bmatrix}
\end{align*}
so using ocatv linsolve I solve this:
\begin{align*}
\begin{bmatrix}
4 & -2 & -2  \\
-2 & 8 & -3 \\
-2 & -3 & 8 \\
0 & -3 & -3 \\
\end{bmatrix} 
* 
\begin{bmatrix}
x_1  \\
x_2 \\
x_3 \\
\end{bmatrix} 
=
\begin{bmatrix}
f_1  \\
f_2 \\
f_3 \\
f_4
\end{bmatrix} 
\end{align*}
After using octave I get $\mybf{x} = (\frac{5}{12}, \frac{2}{12}, \frac{2}{12}, 0)$ as ONE of the possible solutions.
And then $\mybf{y}=-A*\mybf{x} = (\frac{3}{12}, \frac{-7}{12}, \frac{0}{12}, \frac{2}{12}, \frac{2}{12})$
\subsection*{Section 10.1, question 17 (former(Section 8.1, question 17)}
\begin{enumerate}[label=\alph*]
\item 8, because with 9 nodes, a tree that spans is 8 and beyond that we get loops.
\item sum of elements in $\mybf{f}$ is 0.
\item $(A^TA)_{i,i}$ is the number of edges entering node $i$.
\end{enumerate}
\end{document}